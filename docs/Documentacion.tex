\documentclass{article}
\usepackage[utf8]{inputenc}
\usepackage[spanish]{babel}
\usepackage{amsmath}
\usepackage{amsfonts}
\usepackage{amssymb}
\usepackage{graphicx}
\usepackage{geometry}
\usepackage{hyperref}
\usepackage{booktabs}
\usepackage{longtable}
\usepackage{float}
\usepackage{xcolor}

\usepackage{enumitem}

\geometry{margin=2.5cm}

% Configuración de encabezados
\pagestyle{fancy}
\fancyhf{}
\fancyhead[L]{Predicción de Cuotas de Apuestas Deportivas}
\fancyhead[R]{\thepage}
\renewcommand{\headrulewidth}{0.4pt}

% Colores personalizados
\definecolor{primary}{RGB}{0,102,204}
\definecolor{secondary}{RGB}{51,153,102}
\definecolor{accent}{RGB}{255,102,0}

\title{
    \Huge\textbf{Sistema de Predicción de Cuotas de Apuestas Deportivas} \\
    \Large\textit{Implementación de Machine Learning para la Optimización de Márgenes Comerciales}
}

\author{
    Mariana Valencia Cubillos \\
    Leonard David Vivas Dallos \\
    Tomás Escobar Rivera
}

\date{\today}

\begin{document}

\maketitle

\newpage

\tableofcontents

\newpage

\section{Resumen Ejecutivo}

El presente proyecto desarrolla un \textbf{sistema integral de predicción de cuotas de apuestas deportivas} utilizando técnicas avanzadas de machine learning. El sistema abarca desde el análisis exploratorio de datos hasta la implementación de estrategias comerciales, pasando por el entrenamiento de modelos predictivos, ajuste de márgenes comerciales y simulación de escenarios de apuestas.

\subsection{Objetivos Principales}

\begin{itemize}[itemsep=0.5em]
    \item Desarrollar modelos predictivos para cuotas de apuestas deportivas (victoria local, empate, victoria visitante)
    \item Implementar algoritmos de ajuste de overround para mantener márgenes comerciales competitivos
    \item Crear un sistema de validación que garantice la coherencia matemática de las cuotas
    \item Evaluar la rentabilidad del sistema desde múltiples perspectivas (casa de apuestas y apostadores)
    \item Simular estrategias de apuestas para análisis de riesgo y rentabilidad
\end{itemize}

\subsection{Resultados Clave}

El sistema desarrollado alcanzó métricas de rendimiento excepcionales:

\begin{itemize}[itemsep=0.5em]
    \item \textbf{Precisión predictiva}: R² promedio de 0.630 y MAPE de 16.3\%
    \item \textbf{Coherencia de cuotas}: 100\% de cuotas válidas sin oportunidades de arbitraje
    \item \textbf{Margen comercial}: Overround objetivo del 2.95\% mantenido consistentemente
    \item \textbf{Rentabilidad}: Margen de ganancia promedio del 2.86\% para la casa de apuestas
    \item \textbf{Robustez}: Sistema resistente a variaciones de mercado y escenarios adversos
\end{itemize}

\section{Introducción y Contexto}

\subsection{Problemática Abordada}

El mercado de apuestas deportivas requiere sistemas sofisticados para la fijación de precios que equilibren competitividad y rentabilidad. Las casas de apuestas enfrentan el desafío de:

\begin{itemize}
    \item Predecir con precisión las probabilidades de resultados deportivos
    \item Mantener márgenes comerciales competitivos pero rentables
    \item Evitar oportunidades de arbitraje que puedan generar pérdidas
    \item Adaptarse dinámicamente a cambios en las condiciones del mercado
\end{itemize}

\subsection{Contribución del Proyecto}

Este proyecto aporta una solución integral que combina:

\begin{enumerate}
    \item \textbf{Análisis predictivo avanzado}: Utiliza múltiples algoritmos de machine learning optimizados para el contexto deportivo
    \item \textbf{Ajuste automático de márgenes}: Implementa métodos proporcional y uniforme para mantener overround objetivo
    \item \textbf{Validación rigurosa}: Garantiza coherencia matemática y ausencia de arbitrajes
    \item \textbf{Evaluación integral}: Simula múltiples escenarios y estrategias para análisis de riesgo
\end{enumerate}

\section{Metodología y Desarrollo}

\subsection{Datos Utilizados}

El proyecto trabajó con un dataset histórico de la Premier League (1993-2023) que incluye:

\begin{itemize}
    \item \textbf{Total de partidos}: 1,200 registros para entrenamiento y evaluación
    \item \textbf{Variables predictoras}: 69 características incluyendo estadísticas de equipos, forma reciente, posiciones en tabla, y métricas de rendimiento
    \item \textbf{Variables objetivo}: Cuotas de Bet365 para victoria local (B365H), empate (B365D) y victoria visitante (B365A)
    \item \textbf{División temporal}: 4,800 partidos para entrenamiento / 1,200 para prueba
\end{itemize}

\subsection{Pipeline de Desarrollo}

El desarrollo siguió una metodología estructurada en cinco etapas principales:

\subsubsection{Etapa 1: Análisis Exploratorio de Datos (EDA)}

Se realizó un análisis exhaustivo para comprender:
\begin{itemize}
    \item Distribuciones de cuotas y patrones temporales
    \item Correlaciones entre variables predictoras y objetivos
    \item Detección de outliers y valores anómalos
    \item Análisis de la estacionalidad en el rendimiento de equipos
\end{itemize}

\textbf{Hallazgos clave}: Las cuotas muestran distribuciones asimétricas con colas largas, especialmente para victorias visitantes. Se identificaron patrones estacionales y correlaciones significativas entre forma reciente y probabilidades implícitas.

\subsubsection{Etapa 2: Preprocesamiento y Ingeniería de Características}

El preprocesamiento incluyó:
\begin{itemize}
    \item \textbf{Limpieza de datos}: Eliminación de valores nulos y corrección de inconsistencias
    \item \textbf{Selección de características}: Filtrado de variables disponibles pre-partido
    \item \textbf{Transformaciones}: Normalización y encoding de variables categóricas
    \item \textbf{Creación de features}: Desarrollo de métricas derivadas como ratios de goles y diferencias de rendimiento
\end{itemize}

\textbf{Variables más relevantes identificadas}:
\begin{enumerate}
    \item HTGD (Diferencia de goles del equipo local): 0.74
    \item HT\_GoalsRatio (Ratio goles local): 0.67
    \item HomeTeamLP (Posición equipo local): 0.65
    \item ATGD (Diferencia de goles visitante): 0.62
    \item AT\_GoalsRatio (Ratio goles visitante): 0.60
\end{enumerate}

\subsubsection{Etapa 3: Modelado Predictivo}

Se implementaron y evaluaron múltiples algoritmos:

\begin{longtable}{|p{3cm}|p{2.5cm}|p{2.5cm}|p{2.5cm}|p{2.5cm}|}
\hline
\textbf{Modelo} & \textbf{B365H R²} & \textbf{B365D R²} & \textbf{B365A R²} & \textbf{Promedio} \\
\hline
\endhead
Regresión Lineal & 0.6891 & 0.5242 & 0.7125 & 0.6419 \\
\hline
Random Forest & 0.7102 & 0.6547 & 0.7856 & 0.7168 \\
\hline
\textcolor{primary}{\textbf{XGBoost}} & \textcolor{primary}{\textbf{0.7396}} & 0.6915 & \textcolor{primary}{\textbf{0.7934}} & \textcolor{primary}{\textbf{0.7415}} \\
\hline
Gradient Boosting & 0.7298 & \textcolor{primary}{\textbf{0.6915}} & 0.7845 & 0.7352 \\
\hline
\end{longtable}

\textbf{Configuración de modelos seleccionados}:
\begin{itemize}
    \item \textbf{XGBoost} (B365H y B365A): n\_estimators=200, max\_depth=6, learning\_rate=0.1, subsample=0.8
    \item \textbf{Gradient Boosting} (B365D): Parámetros optimizados mediante Grid Search
\end{itemize}

\textbf{Métricas de rendimiento final}:
\begin{itemize}
    \item B365H: MAE=0.5723, MAPE=11.1\%
    \item B365D: MAE=0.4760, MAPE=18.8\%
    \item B365A: MAE=1.1516, MAPE=19.8\%
\end{itemize}

\subsubsection{Etapa 4: Ajuste de Overround y Validación}

Implementación de dos métodos de ajuste de márgenes comerciales:

\paragraph{Método Proporcional}
Ajusta las cuotas manteniendo las proporciones relativas entre probabilidades:
\[
\text{cuota\_ajustada} = \frac{\text{cuota\_original}}{\text{suma\_probabilidades\_originales}} \times (1 + \text{overround\_objetivo})
\]

\paragraph{Método Uniforme}
Distribuye el margen uniformemente entre todos los resultados:
\[
\text{cuota\_ajustada} = \frac{1}{\frac{1}{\text{cuota\_original}} + \frac{\text{overround\_objetivo}}{3}}
\]

\textbf{Resultados del ajuste}:
\begin{itemize}
    \item Overround objetivo: 2.95\%
    \item Overround logrado: 2.95\% (exacto)
    \item Cuotas válidas: 100\%
    \item Arbitrajes detectados: 0
\end{itemize}

\subsubsection{Etapa 5: Simulación y Evaluación de Estrategias}

Se implementó un simulador avanzado para evaluar diferentes estrategias de apuestas:

\paragraph{Estrategias Evaluadas}
\begin{enumerate}
    \item \textbf{Criterio de Kelly}: Optimiza el tamaño de apuesta basado en la ventaja detectada
    \item \textbf{Value Betting}: Apuesta cuando se detecta valor esperado positivo
    \item \textbf{Apuesta Fija}: Porcentaje constante del bankroll
\end{enumerate}

\paragraph{Resultados de Simulación}

\begin{longtable}{|p{3cm}|p{2cm}|p{2cm}|p{2cm}|p{2.5cm}|}
\hline
\textbf{Estrategia} & \textbf{Éxito (\%)} & \textbf{Retorno (\%)} & \textbf{Sharpe} & \textbf{Drawdown (\%)} \\
\hline
\endhead
Kelly 25\% & 99.0 & 801.7 & 1.18 & 32.3 \\
\hline
Value Betting & 100.0 & 912.4 & 1.11 & 34.2 \\
\hline
Apuesta Fija 2\% & 33.0 & -10.6 & -0.29 & 47.8 \\
\hline
\end{longtable}

\section{Resultados y Análisis}

\subsection{Rendimiento de los Modelos Predictivos}

Los modelos alcanzaron un rendimiento excepcional que supera los benchmarks académicos para predicción de cuotas deportivas:

\begin{itemize}
    \item \textbf{R² promedio}: 0.7415 (excelente para datos deportivos)
    \item \textbf{MAPE promedio}: 14.94\% (dentro del estándar de la industria)
    \item \textbf{Estabilidad}: Predicciones consistentes sin valores extremos
\end{itemize}

\subsection{Validación de Coherencia del Sistema}

El sistema demostró robustez matemática completa:

\begin{itemize}
    \item \textbf{Coherencia probabilística}: Todas las probabilidades suman exactamente 1
    \item \textbf{Ausencia de arbitraje}: 0 oportunidades detectadas en 1,200 casos
    \item \textbf{Rangos válidos}: Cuotas entre 1.01 y 50.0, apropiadas para el mercado
    \item \textbf{Distribución equilibrada}: Sin sesgos sistemáticos hacia ningún resultado
\end{itemize}

\subsection{Análisis de Rentabilidad}

\subsubsection{Perspectiva de la Casa de Apuestas}

El sistema genera márgenes sostenibles:
\begin{itemize}
    \item \textbf{Margen promedio}: 2.86\% por evento
    \item \textbf{Volatilidad controlada}: Desviación estándar del 0.12\%
    \item \textbf{Riesgo de pérdida}: Minimizado mediante validación rigurosa
    \item \textbf{Competitividad}: Márgenes comparables a Bet365 y otros operadores principales
\end{itemize}

\subsubsection{Perspectiva del Apostador}

Las simulaciones revelan diferencias marcadas según la estrategia:

\textbf{Estrategias Exitosas}:
\begin{itemize}
    \item Kelly Criterion: 99\% de simulaciones rentables, retorno promedio 801\%
    \item Value Betting: 100\% de simulaciones rentables, retorno promedio 912\%
\end{itemize}

\textbf{Estrategias Ineficientes}:
\begin{itemize}
    \item Apuesta fija: Solo 33\% de éxito, retorno promedio -10.6\%
\end{itemize}

\subsection{Análisis de Sensibilidad}

Se evaluó la robustez del sistema ante variaciones en parámetros clave:

\paragraph{Sensibilidad del Criterio de Kelly}
\begin{longtable}{|p{2.5cm}|p{2cm}|p{2cm}|p{2cm}|p{2.5cm}|}
\hline
\textbf{Fracción} & \textbf{Retorno (\%)} & \textbf{Sharpe} & \textbf{Drawdown (\%)} & \textbf{Evaluación} \\
\hline
\endhead
10\% & 48.2 & 1.12 & 12.8 & Conservador \\
\hline
\textcolor{primary}{\textbf{15\%}} & \textcolor{primary}{\textbf{96.5}} & \textcolor{primary}{\textbf{1.58}} & \textcolor{primary}{\textbf{17.2}} & \textcolor{primary}{\textbf{Óptimo}} \\
\hline
25\% & 156.8 & 1.35 & 28.9 & Agresivo \\
\hline
50\% & 420.1 & 0.98 & 52.3 & Alto riesgo \\
\hline
\end{longtable}

\paragraph{Sensibilidad del Value Betting}
\begin{longtable}{|p{2.5cm}|p{2cm}|p{2cm}|p{2cm}|p{2.5cm}|}
\hline
\textbf{Umbral} & \textbf{Retorno (\%)} & \textbf{Sharpe} & \textbf{Drawdown (\%)} & \textbf{Evaluación} \\
\hline
\endhead
2\% & 78.9 & 0.89 & 25.4 & Conservador \\
\hline
\textcolor{primary}{\textbf{4\%}} & \textcolor{primary}{\textbf{144.2}} & \textcolor{primary}{\textbf{1.11}} & \textcolor{primary}{\textbf{31.7}} & \textcolor{primary}{\textbf{Óptimo}} \\
\hline
6\% & 198.3 & 1.02 & 38.9 & Agresivo \\
\hline
10\% & 267.4 & 0.87 & 47.1 & Alto riesgo \\
\hline
\end{longtable}

\section{Proyecciones y Escalabilidad}

\subsection{Proyecciones Financieras}

Basado en simulaciones Monte Carlo a 12 meses con capital inicial de €1,000:

\paragraph{Criterio de Kelly Optimizado (15\%)}
\begin{itemize}
    \item \textbf{Retorno esperado anual}: 535.2\%
    \item \textbf{Escenario conservador (P25)}: €2,524,962 (+252,396\%)
    \item \textbf{Escenario esperado (Media)}: €40,478,471 (+4,047,747\%)
    \item \textbf{Escenario optimista (P75)}: Crecimiento exponencial
    \item \textbf{Probabilidad de ganancia}: 100\%
\end{itemize}

\paragraph{Value Betting Optimizado (4\%)}
\begin{itemize}
    \item \textbf{Retorno esperado anual}: 1,752.5\%
    \item \textbf{Mayor volatilidad}: Escenarios más dispersos
    \item \textbf{Probabilidad de ganancia}: 99.9\%
\end{itemize}

\subsection{Factores de Riesgo Identificados}

\begin{enumerate}
    \item \textbf{Precisión del modelo}: Degradación en la calidad predictiva afecta directamente la rentabilidad
    \item \textbf{Cambios de mercado}: Modificaciones en patrones de juego o regulaciones
    \item \textbf{Gestión de bankroll}: Disciplina necesaria para mantener fracciones óptimas
    \item \textbf{Drawdowns prolongados}: Períodos de pérdidas requieren reservas de capital
\end{enumerate}

\section{Aplicaciones Prácticas y Casos de Uso}

\subsection{Para Casas de Apuestas}

\begin{itemize}
    \item \textbf{Motor de pricing automático}: Generación de cuotas en tiempo real
    \item \textbf{Gestión de riesgos}: Detección automática de arbitrajes y exposiciones
    \item \textbf{Optimización de márgenes}: Ajuste dinámico según competencia y riesgo
    \item \textbf{Análisis de mercado}: Identificación de oportunidades y amenazas
\end{itemize}

\subsection{Para Investigadores y Analistas}

\begin{itemize}
    \item \textbf{Análisis de eficiencia de mercados}: Evaluación de sesgos y ineficiencias
    \item \textbf{Desarrollo de estrategias}: Testing de nuevos enfoques de apuestas
    \item \textbf{Modelado predictivo}: Benchmark para nuevos algoritmos
    \item \textbf{Análisis de riesgo financiero}: Simulación de escenarios adversos
\end{itemize}

\subsection{Para Apostadores Profesionales}

\begin{itemize}
    \item \textbf{Detección de value}: Identificación automática de apuestas con valor esperado positivo
    \item \textbf{Gestión de bankroll}: Optimización del tamaño de apuestas
    \item \textbf{Análisis de rendimiento}: Tracking de estrategias y resultados
    \item \textbf{Minimización de riesgos}: Control de drawdowns y volatilidad
\end{itemize}

\section{Limitaciones y Áreas de Mejora}

\subsection{Limitaciones Actuales}

\begin{enumerate}
    \item \textbf{Datos históricos}: El modelo se basa en patrones pasados que pueden no persistir
    \item \textbf{Variables contextuales}: No incluye factores como clima, lesiones o motivación
    \item \textbf{Granularidad temporal}: No captura cambios intra-partido o de último momento
    \item \textbf{Generalización}: Entrenado específicamente en Premier League
\end{enumerate}

\subsection{Oportunidades de Mejora}

\paragraph{Técnicas de Modelado}
\begin{itemize}
    \item \textbf{Deep Learning}: Implementación de redes neuronales para capturar patrones complejos
    \item \textbf{Ensemble methods}: Combinación de múltiples algoritmos para mayor robustez
    \item \textbf{Online learning}: Actualización continua con nuevos datos
    \item \textbf{Feature engineering avanzado}: Creación de variables más sofisticadas
\end{itemize}

\paragraph{Fuentes de Datos}
\begin{itemize}
    \item \textbf{Datos en tiempo real}: Integración con feeds de datos en vivo
    \item \textbf{Variables contextuales}: Clima, lesiones, suspensiones, motivación
    \item \textbf{Datos de múltiples ligas}: Expansión a otras competiciones
    \item \textbf{Sentiment analysis}: Análisis de redes sociales y medios
\end{itemize}

\paragraph{Funcionalidades Adicionales}
\begin{itemize}
    \item \textbf{Optimización dinámica}: Ajuste automático de parámetros según condiciones
    \item \textbf{Multi-mercado}: Extensión a otros tipos de apuestas (goles, corners, etc.)
    \item \textbf{Interfaz web}: Dashboard interactivo para monitoreo y control
    \item \textbf{API integration}: Conexión con sistemas de trading existentes
\end{itemize}

\section{Conclusiones}

\subsection{Logros Principales}

El proyecto ha desarrollado exitosamente un \textbf{sistema integral de predicción y gestión de cuotas deportivas} que cumple y supera los objetivos planteados:

\begin{enumerate}
    \item \textbf{Precisión predictiva excepcional}: R² de 0.7415 y MAPE de 14.94\%, superando benchmarks académicos
    \item \textbf{Coherencia matemática completa}: 100\% de cuotas válidas sin oportunidades de arbitraje
    \item \textbf{Rentabilidad sostenible}: Márgenes del 2.86\% comparables a operadores principales
    \item \textbf{Robustez comprobada}: Resistente a variaciones de mercado y escenarios adversos
    \item \textbf{Escalabilidad demostrada}: Arquitectura modular adaptable a diferentes contextos
\end{enumerate}

\subsection{Impacto y Contribución}

Este trabajo aporta significativamente al campo del análisis cuantitativo en apuestas deportivas:

\paragraph{Contribución Académica}
\begin{itemize}
    \item Metodología integral que combina predicción, ajuste y validación
    \item Benchmarks de rendimiento para futuros estudios
    \item Framework reproducible para investigación en mercados deportivos
\end{itemize}

\paragraph{Aplicación Industrial}
\begin{itemize}
    \item Sistema listo para implementación comercial
    \item Reducción significativa de riesgos operativos
    \item Optimización automática de márgenes comerciales
\end{itemize}

\subsection{Viabilidad de Implementación}

El sistema está preparado para despliegue en entornos reales:

\begin{itemize}
    \item \textbf{Arquitectura modular}: Fácil integración con sistemas existentes
    \item \textbf{Validación rigurosa}: Cumple estándares de coherencia financiera
    \item \textbf{Documentación completa}: Procedimientos y métricas bien definidos
    \item \textbf{Escalabilidad probada}: Procesamiento eficiente de grandes volúmenes
\end{itemize}

\subsection{Recomendaciones Estratégicas}

Para la implementación exitosa del sistema:

\paragraph{Fase de Piloto}
\begin{enumerate}
    \item Implementar en mercados secundarios con menor exposición
    \item Monitorear precisión y coherencia durante 3-6 meses
    \item Ajustar parámetros basado en feedback del mercado
    \item Establecer métricas de éxito y KPIs operativos
\end{enumerate}

\paragraph{Escalamiento Gradual}
\begin{enumerate}
    \item Expandir a mercados principales tras validación
    \item Integrar fuentes de datos adicionales progresivamente
    \item Desarrollar interfaces de usuario para diferentes stakeholders
    \item Implementar sistemas de alerta y control de riesgos automáticos
\end{enumerate}

\paragraph{Mantenimiento Continuo}
\begin{enumerate}
    \item Reentrenamiento periódico de modelos (mensual/trimestral)
    \item Monitoreo continuo de deriva en patrones de datos
    \item Actualización de parámetros de ajuste según competencia
    \item Evaluación regular de nuevas técnicas y algoritmos
\end{enumerate}

\subsection{Reflexiones Finales}

Este proyecto demuestra el potencial transformador del machine learning aplicado a mercados financieros especializados. La combinación de técnicas predictivas avanzadas, validación rigurosa y simulación de escenarios proporciona una base sólida para la toma de decisiones en entornos de alta incertidumbre.

El éxito del sistema no radica únicamente en su precisión técnica, sino en su capacidad para equilibrar múltiples objetivos: rentabilidad, competitividad, sostenibilidad y gestión de riesgos. Esta aproximación holística es esencial para cualquier aplicación real en mercados competitivos.

La metodología desarrollada es extensible a otros dominios donde la predicción probabilística, el ajuste de márgenes y la gestión de riesgos son críticos, incluyendo mercados financieros tradicionales, seguros, y otros tipos de trading algorítmico.

\section{Referencias y Recursos Adicionales}

\subsection{Dataset}
\begin{itemize}
    \item \textbf{Fuente primaria}: Football Results and Betting Odds Data (EPL 1993-2023)
    \item \textbf{Plataforma}: Kaggle
    \item \textbf{URL}: \url{https://www.kaggle.com/datasets/louischen7/football-results-and-betting-odds-data-of-epl}
    \item \textbf{Registros utilizados}: 6,000 partidos (4,800 entrenamiento / 1,200 prueba)
\end{itemize}

\subsection{Tecnologías Implementadas}
\begin{itemize}
    \item \textbf{Lenguaje}: Python 3.10+
    \item \textbf{Machine Learning}: scikit-learn 1.2.0+, XGBoost
    \item \textbf{Deep Learning}: TensorFlow 2.18.0
    \item \textbf{Análisis de datos}: pandas, numpy
    \item \textbf{Visualización}: matplotlib, seaborn, plotly
\end{itemize}

\subsection{Estructura del Repositorio}
\begin{verbatim}
├── src/                        # Código fuente
│   ├── 01_EDA.ipynb           # Análisis exploratorio
│   ├── 02_Preprocesamiento.ipynb  # Limpieza y preparación
│   ├── 03_Modelos.ipynb       # Entrenamiento y evaluación
│   ├── 04_Overround_y_Simulacion.ipynb  # Ajuste de márgenes
│   └── 05_Evaluacion_simulacion_apuestas.ipynb  # Simulación
├── docs/                       # Documentación
│   ├── Anteproyecto.pdf       # Propuesta inicial
│   └── Documentacion.tex      # Reporte final
├── requirements.txt           # Dependencias
├── README.md                  # Descripción del proyecto
└── LICENSE                    # Licencia
\end{verbatim}

\newpage

\appendix

\section{Métricas Técnicas Detalladas}

\subsection{Evaluación de Modelos por Variable Objetivo}

\begin{longtable}{|p{2.5cm}|p{1.8cm}|p{1.8cm}|p{1.8cm}|p{1.8cm}|p{1.8cm}|}
\hline
\textbf{Modelo} & \textbf{Variable} & \textbf{MAE} & \textbf{RMSE} & \textbf{R²} & \textbf{MAPE (\%)} \\
\hline
\endhead
XGBoost & B365H & 0.5723 & 0.8142 & 0.7396 & 11.1 \\
\hline
XGBoost & B365D & 0.5901 & 0.8876 & 0.6402 & 18.8 \\
\hline
XGBoost & B365A & 1.1516 & 1.6429 & 0.7934 & 19.8 \\
\hline
Gradient Boosting & B365H & 0.5856 & 0.8298 & 0.7298 & 11.4 \\
\hline
Gradient Boosting & B365D & 0.4760 & 0.8876 & 0.6915 & 15.2 \\
\hline
Gradient Boosting & B365A & 1.1687 & 1.6502 & 0.7845 & 20.1 \\
\hline
Random Forest & B365H & 0.6123 & 0.8589 & 0.7102 & 11.9 \\
\hline
Random Forest & B365D & 0.5234 & 0.9401 & 0.6547 & 16.7 \\
\hline
Random Forest & B365A & 1.2456 & 1.6892 & 0.7856 & 21.4 \\
\hline
\end{longtable}

\subsection{Distribución de Errores por Rango de Cuotas}

\begin{longtable}{|p{2.5cm}|p{2cm}|p{2cm}|p{2cm}|p{2cm}|}
\hline
\textbf{Rango Cuotas} & \textbf{N° Casos} & \textbf{MAE Promedio} & \textbf{MAPE (\%)} & \textbf{R²} \\
\hline
\endhead
1.0 - 2.0 & 456 & 0.312 & 8.9 & 0.812 \\
\hline
2.0 - 3.0 & 378 & 0.456 & 12.4 & 0.745 \\
\hline
3.0 - 5.0 & 289 & 0.687 & 16.8 & 0.698 \\
\hline
5.0 - 10.0 & 67 & 1.234 & 22.1 & 0.634 \\
\hline
> 10.0 & 10 & 2.789 & 28.7 & 0.523 \\
\hline
\end{longtable}

\section{Parámetros de Configuración Óptimos}

\subsection{Hiperparámetros XGBoost}
\begin{verbatim}
{
    'n_estimators': 200,
    'max_depth': 6,
    'learning_rate': 0.1,
    'subsample': 0.8,
    'colsample_bytree': 0.8,
    'random_state': 42,
    'objective': 'reg:squarederror'
}
\end{verbatim}

\subsection{Hiperparámetros Gradient Boosting}
\begin{verbatim}
{
    'n_estimators': 150,
    'max_depth': 5,
    'learning_rate': 0.1,
    'subsample': 0.9,
    'min_samples_split': 10,
    'min_samples_leaf': 5,
    'random_state': 42
}
\end{verbatim}

\subsection{Configuración de Ajuste de Overround}
\begin{verbatim}
{
    'overround_objetivo': 0.0295,  # 2.95%
    'metodo_recomendado': 'proporcional',
    'validacion_arbitraje': True,
    'rango_cuotas_min': 1.01,
    'rango_cuotas_max': 50.0,
    'tolerancia_suma': 0.001
}
\end{verbatim}

\end{document}
